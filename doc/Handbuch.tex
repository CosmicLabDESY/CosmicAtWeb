% -*- TeX:DE:UTF-8 -*-
\documentclass[version=last,
	a4paper,			% alle weiteren Papierformat einstellbar
	pagesize, 			% Diese Option reicht die Papiergröße an alle Ausgabeformate weiter
	11pt,				% Schriftgröße
	BCOR1cm,			% Bindekorrektur, bspw. 1 cm
	DIV12,	 			% Errechnet einen guten Satzspiegel
	pointlessnumbers,   % Überschriftennummerierung ohne abschließenden Punkt
%	smallheadings,		% kleinere Überschriften
	halfparskip,		% Absätze nicht einrücken, sondern Abstand davor
%	draft				% Entwurfsansicht, überlangen Zeilen in Ausgabe gekennzeichnet
]{scrreprt}
\usepackage[utf8]{inputenc}
\usepackage[T1]{fontenc}
\usepackage[english,ngerman]{babel}
\usepackage[strict,style=english,autostyle=true]{csquotes} \MakeOuterQuote{"}  % Automatische
\usepackage{marvosym}
\usepackage{textcomp}
\usepackage{lmodern}
%\renewcommand*\familydefault{\sfdefault} %% base font is to be sans serif
\usepackage{scrpage2}
\usepackage{booktabs}
\usepackage{icomma}
\usepackage{siunitx}
\usepackage{microtype}
\usepackage{graphicx}
\usepackage{float}
\usepackage{fixltx2e}
\usepackage{lastpage}
\usepackage{hyperref}
\hypersetup{pdfpagelabels=true,pdfborder={0 0 0},breaklinks=true,
            bookmarksnumbered=true,pdfpagelayout=OneColumn,pdfstartview={XYZ null null 1},bookmarks=false}
\usepackage{xcolor}
\usepackage{xspace}
\usepackage{listings}

%% Satzenginetuning %%%%%%%%%%%%%%%%%%%%%%%%%%%%%%%%%%%%%%%%%%%%%%%%%%%%%%%%%%
\clubpenalty = 10000 \widowpenalty = 10000 \displaywidowpenalty = 10000


\lstset{
   basicstyle=\footnotesize\ttfamily,
%   numbers=right,
   numberstyle=\tiny,
	breaklines=true, breakautoindent=true,
%	columns=flexible,
	frame=single,
%	backgroundcolor=\color{lightgray},
%	xleftmargin=1em,
%	xrightmargin=1em,
	captionpos=b,
	aboveskip=2ex
}

\newcommand{\zb}{z.\,B.\xspace}

\begin{document}

%\extratitle{extra}
%\titlehead{Kopf}
%\subject{Typisierung}
\title{\texttt{pyplot}-Handbuch}
%\subtitle{Untertitel}
\author{Adam Lucke}
\date{\today}
%\publishers{Herausgeber}
%\thanks{Fußnote}
\maketitle

\tableofcontents
\chapter{Installation}
Eine Installation des Softwarepaketes ist eigentlich nicht notwendig. Unter Installation ist hier zu verstehen, dass das Paket \texttt{pyplot.zip} in das gewünschte Verzeichnis entpackt wird.

Die Software ist in Python geschrieben und sollte daher auf allen Systemen laufen, auf denen Python und die notwendigen Bibliotheken installiert sind.

\section{Bibliotheken}
Im folgenden sind alle notwendigen Bibliotheken aufgelistet, die für \texttt{pyplot} notwendig sind.
\begin{itemize}
  \item Python 2.7+, \url{http://python.org/}
  \item pytz, \url{http://pytz.sourceforge.net/}
  \item dateutil 1.5+, \url{http://labix.org/python-dateutil}
  \item Numpy 0.9+/Scipy 1.6+, \url{https://www.scipy.org}
  \item pytables 2.2+ (mit zlib), \url{http://pytables.github.com}
  \item numexpr 1.4+ (für pytables), \url{http://code.google.com/p/numexpr/}
  \item matplotlib 1.1+, \url{http://matplotlib.sourceforge.net/}
  \item basemap 1.0+, \url{http://www.scipy.org/Cookbook/Matplotlib/Maps}
  \item PIL 1.1+ (für basemap), \url{http://www.pythonware.com/products/pil/}
\end{itemize}

Sind diese Bibliotheken installiert, so können die \texttt{pyplot}-Tools verwendet werden.


\chapter{Rohdatenkonvertierung}
Die Aufbereitung der Rohdaten von verschiedenen Experimenten erfolgt in zwei Schritten. Es wird davon ausgegangen, dass die Rohdaten in Form von ASCII-Dateien vorliegen. Zu jedem Experiment (Detektor) gehören Detektor-Events und Zusatzdaten. Dies sind \zb die Event-Dateien des CosmicTrigger ($\mu$-Hodoskop) und die dazugehörigen Wetterdaten der Wetterstation im DESY Zeuthen. Beide Daten liegen als ASCII-Dateien vor und haben die Struktur einer Tabelle.

\begin{lstlisting}[label=lst:ctascii,caption={Beispiel von CosmicTrigger-Rohdaten}]
2011-Jan-01 00:00:07.544454 CET+1   0 1 0 0    0 1 0 0    1 0 0 0
2011-Jan-01 00:00:12.396551 CET+1   0 1 0 1    1 1 0 0    1 0 1 1
2011-Jan-01 00:00:17.764563 CET+1   1 0 0 0    1 0 0 0    0 0 1 0
2011-Jan-01 00:00:31.580567 CET+1   0 0 1 0    0 0 1 0    0 0 0 1
\end{lstlisting}

\begin{lstlisting}[label=lst:zweather,caption={Beispiel von Zeuthen-Wetter-Rohdaten},	
                   xleftmargin=-1.5em,xrightmargin=-1.5em]
2011-01-01 00:30:00+01:00 16.9 1.2 -0.1 33 91 0.4 22.5  NNE -1.0 1.2 0.0 1012.9
2011-01-01 01:00:00+01:00 16.8 1.1 -0.2 33 91 0.3 0.0   N   -1.0 1.1 0.0 1012.4
2011-01-01 01:30:00+01:00 16.8 1.1 -0.4 33 90 1.0 22.5  NNE -1.0 1.1 0.0 1012.0
2011-01-01 02:00:00+01:00 16.8 1.1 -0.4 33 90 0.7 337.5 NNW -1.0 1.1 0.0 1011.5
\end{lstlisting}

Im \emph{ersten Schritt} werden diese ASCII-Tabellen in  HDF5-Tabellen umgewandelt. HDF5 ist ein binäres Datenformat, das für schnellen Zugriff und effizient Speicherung von \zb tabellarischen Daten optimiert wurde (siehe \url{http://www.hdfgroup.org/HDF5/}). Die Umwandlung von ASCII $\rightarrow$ HDF5 erfolgt mit dem Kommandozeilentool \texttt{rawdata.py}. Bei dieser Umwandlung werden die Daten gleichzeitig auf Gültigkeit überprüft und eventuell fehlerhafte Daten entfernt. Für Eventdaten und Zusatzdaten wird jeweils eine eigene HDF5-Tabelle erzeugt.

Im \emph{zweiten Schritt} werden den Eventdaten die Zusatzdaten durch Interpolation zugeordnet. So wird \zb jedem CosmicTrigger-Event eine Wetterinformation zugeordnet. Zum exakten Zeitpunkt des Events sind aber i.d.R. keine Wetterdaten erfasst worden. Deswegen werden die Wetterdaten zum Zeitpunkt des Events aus den vorhandenen Wetterdaten vor und nach dem Eventzeitpunkt durch Interpolation berechnet. Das Zusammenführen der Event- mit den Zusatzdaten erfolgt mit dem Kommandozeilentool \texttt{merge.py}. Es wird eine neue HDF5-Tabelle erzeugt, die die Events mit interpolierten Zusatzdaten enthält.

Die Rohdaten in HDF5-Format sowie die zusammengeführten ("gemergedten") Daten werden im weiteren zur Erzeugung von grafischen Darstellungen verwendet.

\section{Zeitdarstellung}
\subsection{in ASCII}
Die Variable "Zeit" wird in der Darstellung gegenüber anderen Variablen gesondert behandelt. Größen wie Temperatur oder Spannung können einfach durch eine Zahl dargestellt werden. Für Zeitpunkte hingegen gibt es vielfältige Möglichkeiten der Darstellung. Der 11. September 2001 kann \zb geschrieben werden als \texttt{11.09.2001} oder \texttt{2011/09/11}. Eine Angabe von Datum und Zeit \zb 11. September 2001, 9:30 Uhr kann geschrieben werden als \texttt{09:30 11.09.2001} oder \texttt{2011/09/11 09:30}. Beide Formate sind unterschiedlich und es gibt noch viel mehr mögliche Formate, um einen Zeitpunkt darzustellen. Zudem fehlt beiden Formaten die \emph{Angabe der Zeitzone}. Es ist nicht klar, ob 09:30 in Berlin, Honkong oder New York gemeint ist.

Ohne Angabe der Zeitzone sind Zeitangaben nicht vergleichbar. Besondere Vorsicht ist bei Sommer- und Winterzeit geboten. Es genügt also nicht anzugeben, dass es sich \zb um Berliner Lokalzeit handelt, es ist auch anzugeben ob und wie zwischen Sommer- und Winterzeit umgestellt wird. Es ist ratsam Zeitangaben grundsätzlich in UTC zu machen, was viele Probleme vermeidet und auch Schaltsekunden korrekt berücksichtigt. (Rechen mit Zeiten ist keine triviales Problem, auch wenn es auf den ersten Blick so scheint, und ist in vielen Bibliotheken fehlerhaft implementiert.)

Um Zeitangaben automatisiert verarbeiten zu können, müssen Zeitangaben (in ACSII-Dateien) den folgenden Konventionen genügen:
\begin{itemize}
  \item Die Darstellung erfolgt im \emph{ISO-8601-Format} (siehe \url{http://de.wikipedia.org/wiki/ISO_8601}), leichte Abweichungen sind zulässig\footnote{wenn sie von \texttt{dateutil.parser} (\url{http://labix.org/python-dateutil}) verstanden werden}.
  \item Alle Zeitangaben müssen eine \emph{Zeitzoneninformation} enthalten bzw. muss bekannt sein in welcher Zeitzone die Zeitangaben zu interpretieren sind.
\end{itemize}

\subsection{in HDF5}
Die in den Rohdaten als ISO-String kodierten Zeitangaben werden bei der Konvertierung nach HDF5 in Zahlen (double) umgewandelt. Es wird \zb $t=$\texttt{2001-11-09T09:30:00+0100} zu $-67621004.658$. Die Zahl $-67621004.658$ ist Anzahl der Sekunden seit dem Bezugszeitpunkt $t_0=$\texttt{2004-01-01T00:00:00+0000}, also $t-t_0=-67621004.658\si{s}$.

Der Bezugszeitpunkt kann während der Konvertierung frei gewählt werden.

Die Spalte \texttt{time} in der HDF5-Tabelle enthält also  die Anzahl der Sekunden seit dem Bezugszeitpunkt $t_0$. Der Bezugszeitpunkt wird als Tabellenattribut \texttt{t0} in Form eines ISO-Strings mit der Tabelle gespeichert. Dadurch kann einerseits leicht mit den (relativen) Zeiten gerechnet werden, es ist aber auch immer möglich, die "relativen Sekunden" wieder in Zeitpunkte in ISO-Format umzurechnen.

\section{rawdata.py}
\subsection{Anwendung}
Das Konvertieren von Rohdaten nach HDF5 mit \texttt{rawdata.py} erfolgt so: Man ruft \texttt{rawdata.py} auf der Konsole\footnote{für Windows empfiehlt sich \url{http://sourceforge.net/projects/console/}} auf und übergibt die zu konvertierenden Dateien als Parameter.
\begin{lstlisting}
rawdata.py events1.txt events2.txt wetter.txt
\end{lstlisting}
Dies erzeugt die Datei \texttt{out.h5}, welche zwei HDF5-Tabellen (Events und Wetter) enthält.

Weitere Optionen von \texttt{rawdata.py} erhält man über der Parameter \verb"--help":
\begin{lstlisting}[caption={Kommandozeilen-Hilfe von rawdata.py}]
usage: rawdata.py [-h] [--version] [-o file] [-f] [-t datetime] [-s] [-q]
                  infiles [infiles ...]

convert raw ASCII data tables into tables in one HDF5 file

positional arguments:
  infiles               input files, if a directory is given, all files in
                        it and in its subdirectories are used

optional arguments:
  -h, --help            show this help message and exit
  --version             show program's version number and exit
  -o file, --out file   HDF5 output file (default: out.h5)
  -f, --force           overwrite existing file
  -t datetime, --reftime datetime
                        reference time t0 (default: 2004-01-01T00:00:00+0000)
  -s, --noskip          do not skip invalid lines, stop on error
  -q, --quiet           do not show progressbar, just print error messages
\end{lstlisting}

Die einzelnen Optionen bedeuten:
\begin{description}
  \item[\verb"-o"] dient der Angabe der Ausgabedatei. Ist die Option nicht angegeben wird nach \texttt{out.h5} geschrieben.
  \item[\verb"-s"] schaltet das Überspringen von ungültigen Zeilen aus. Schlägt die Überprüfung einer Zeile fehl, so wird die entsprechende Zeile nicht übersprungen, sondern die Konvertierung wird abgebrochen.
  \item[\verb"-f"] erzwingt das Überschreiben einer eventuell vorhandenen Ausgabedatei.
  \item[\verb"-q"] unterdrückt die Anzeige des Fortschrittbalkens.
  \item[\verb"-t"] legt den Zeit-Bezugspunkt fest, Standardwert \texttt{2004-01-01T00:00:00+0000}
\end{description}

Als \texttt{infiles} können beliebige Dateien oder Verzeichnisse angegeben werden. Wird ein Verzeichnis angegeben, so ließt \texttt{rawdata.py} alle Dateien in diesem Verzeichnis und allen Unterverzeichnissen ein. Den Typ der in den Dateien enthaltenen Daten detektiert \texttt{rawdata.py} automatisch. Die Reihenfolge der Angabe der Dateien/Verzeichnisse ist irrelevant, \texttt{rawdata.py} sortiert die Dateien intern nach dem jeweils ersten Zeiteintrag in der Datei.

\subsection{Funktionsweise}
Die Konvertierung der ASCII-Rohdaten zu HDF5-Tabellen mit \texttt{rawdata.py} erfolgt nach folgendem Schema:
\begin{enumerate}
  \item \emph{automatische Detektion des Typs jeder Eingabedatei}\\
        Jede Eingabedatei wird geöffnet und versucht mit allen vorhandenen "Einlesemethoden" zu lesen. Gelingt das Einlesen der ersten 10 Datenzeilen mit nur einer der möglichen Einlesemethoden, so ist der Typ der Datei bestimmt. Alle Dateien gleichen Typs werden in einer jeweils eigenen Liste zusammengefasst.
  \item \emph{Sortieren aller Eingabedateien jeweils gleichen Typs nach Zeit}\\
        Jeder Liste von Dateien gleichen Typs wird nach Zeit sortiert. Dabei wird die jeweils erste Datenzeile jeder Datei gelesen und der Eintrag der Spalte \texttt{time} extrahiert. Auf Basis dieser Zeiteinträge werden die Dateien sortiert.
  \item \emph{zeilenweise Verarbeitung aller Eingabedateien jeweils gleichen Typs}\\
        Für jede Liste von Dateien gleichen Typs wird eine HDF5-Tabelle erstellt. Die sortierten Dateien werden der Reihe nach Zeile für Zeile durchgegangen, konvertiert, bereinigt und verifiziert. Für jeden Datenzeile in den Eingabedateien wird eine Zeile an die HDF5-Tabelle angefügt. Das Interpretieren, Umwandeln der ASCII-Eingaben in die entsprechenden programminternen Datentypen (double, int, etc.) erfolgt durch sog. \emph{LineHandler}. Für jeden Dateityp gibt es einen eigenen LineHandler, der die Daten auch bereinigt und verifiziert.

\end{enumerate}

\subsubsection{LineHandler}
Der LineHandler verarbeitet die Zeilen der Eingabedateien. Die Klassen \texttt{LineHandler} ist in \texttt{rawdata.py} definiert. Implementierungen müssen folgende Methoden/Felder enthalten:
\begin{description}
  \item[\texttt{\_\_call\_\_(self, line)}] Ein LineHandler ist callable, übergeben wird eine Zeile der Eingabedatei (str). Der Rückgabewert ist \texttt{None}, wenn die Zeile ignoriert/übersprungen wird oder ein Tupel der extrahierten Daten, so wie es in die HDF5-Tablle eingetragen wird. Diese Methode wird für jede Zeile der Eingabedateien aufgerufen. Die Eingabedaten sollen hier gelesen, konvertiert und geprüft werden. Bei fehlgeschlagener Prüfung, wird eine Exception ausgelöst.
  \item[\texttt{description}] (str) Beschreibungen des Handlers
  \item[\texttt{table\_name}] (str) Name der zugehörigen HDF5-Tabelle
  \item[\texttt{cols\_and\_units}] (OrderedDict) geordnetes Dictionary der Tabllenspalten und der zugehörigen Einheiten (Bsp. \texttt{cols\_and\_units = OrderedDict([('time', 's'), ...])})
\end{description}

Die Variable \texttt{available\_handlers} in \texttt{rawdata.py} enthält eine Auflistung vorhandender Implementierungen von LineHandlern. Fügt man eigene Implementierungen dieser Liste hinzu, so werden diese von bei der Konvertierung mitverwendet. \texttt{rawdata.py} kann also leicht um zusätzliche Rohdatenformate erweitert werden.

\subsection{vorhandene LineHandler}
Die folgenden LineHandler sind in \texttt{rawdata.py} bereits vorhanden:
\begin{description}
  \item[\texttt{WeatherHandler}] Ließt Wetterdaten der Wetterstation Zeuthen. Das Datenformat hat eine Struktur wie in \autoref{lst:zweather}. Die Felder einer Zeile sind durch Leerzeichen getrennt und enthalten die Werte: 
      \textsf{time [s], T\_i [C°], T\_a [C°], T\_dew [C°], H\_i [\%], H\_a [\%], v\_wind [m/s], d\_wind [°], gust, chill, rain [mm], p [hPa], clouds}
  \item[\texttt{CTEventHandler}] Ließt CosmicTrigger-Events im Format wie in \autoref{lst:ctascii}. Die Felder einer Zeile sind durch Leerzeichen getrennt und enthalten die Werte: \textsf{time [s], a1, a2, a3, a4, b1, b2, b3, b4, c1, c2, c3, c4}. Dabei bedeutet \textsf{a1} = Ebene a, Element 1, usw.
  \item[\texttt{ITTEventHandler}] Ließt Eventdaten des Zeuthener IctTop-Tanks. Es werden immer zwei aufeinander folgende Zeilen gelesen. Die Daten haben die folgende Struktur:
      \begin{lstlisting}
 0 LaCoMe(SD) 2.1 2011/10/24 10:56:33
 0       Run       181    100000         0
 0   Comment
 0 2011/10/24 10:56:33.929  V265[0]  666600  12 0 1
 0 2011/10/24 10:56:33.929  V265[2]  776600  12 0 1 2 3 4 5 6 7
 1 2011/10/24 10:56:33.929  V265[0]  27   9
 1 2011/10/24 10:56:33.929  V265[2]  27  28  25  32  25  32  29  31
 2 2011/10/24 10:56:33.929  V265[0]  40  16
 2 2011/10/24 10:56:33.929  V265[2]  27  28  27  33  28  33  32  33
   ...
-2 2011/10/24 11:34:06.596  V265[0]  40  16
-2 2011/10/24 11:34:06.596  V265[2]  29  29  30  34  30  34  34  35
-1       end    100000 2011/10/24 11:34:06

      \end{lstlisting}
      Von Interesse sind die Zeilen, die mit Zahlen $>0$ beginnen. Darauf folgt ein Zeitstempel (\textsf{time}). Die 2 bzw. 8 Zahlen am Ende der Zeilen sind die gemessenen Ladungen in dem DOMs (\textsf{dom1, dom2}) bzw. den Segmenten der Triggerebenen (\textsf{a1-b4}).
      
      Die Felder der zurückgegebenen Daten sind: \textsf{time [s], dom1, dom2, run, time2 [s], a1, a2, a3, a4, b1, b2, b3, b4}.
      Dabei bedeutet \textsf{a1} = Ebene a, Element 1, usw., \textsf{time2} ist der Zeitstempel der zweiten Zeile.
  \item[\texttt{PolarsternHandler}] Ließt Daten aus vorprozessierten Datenfiles des Myondetektors auf der Polarstern.
\end{description}


\section{merge.py}
Mit diesem Tool werden Event- und Zusatzdaten in HDF5-Format durch Interpolation zusammengeführt. Jedem Event wird ein Satz Zusatzdaten hinzugefügt, welcher durch Interpolation auf den Eventzeitpunkt berechnet wird.

\subsection{Anwendung}
Man ruft \texttt{merge.py} auf Kommandozeile auf:
\begin{lstlisting}
merge.py /raw/events /raw/weather in.h5
\end{lstlisting}
Dies erzeugt eine neue Tabelle in \texttt{in.h5}, welche die Zeilen aus \texttt{table1} mit interpolierten Werten aus \texttt{table2} enthält.

Weitere Optionen von \texttt{merge.py} erhält man über den Parameter \texttt{--help}:
\begin{lstlisting}[caption={Kommandozeilen-Hilfe von rawdata.py}]
usage: merge.py [-h] [--version] [-m column] [-x seconds] [-o file] [-f] [-q]
                primary secondary infile

merge two tables in a HDF5 file

positional arguments:
  primary               name of primary table (event table)
  secondary             name of secondary table (weather table)
  infile                HDF5 file with tables to merge

optional arguments:
  -h, --help            show this help message and exit
  --version             show program's version number and exit
  -m column, --merge column
                        column to merge on (default: time)
  -x seconds, --maxint seconds
                        max. interval length in secondary table (default: 4h)
  -o file, --out file   HDF5 output file (default: write to infile)
  -f, --force           overwrite existing file
  -q, --quiet           do not show progressbar, just print error messages
\end{lstlisting}

Die einzelnen Optionen bedeuten:
\begin{description}
  \item[\verb"-o"] Ausgabedatei (Standard: schreibe in Eingabedatei)
  \item[\verb"-m"] auf Grund welcher Spalte soll gemerget werden (Standard: \texttt{time})
  \item[\verb"-x"] maximal Zeit (Sekunden) zwischen zwei Zeilen der Sekundärtabelle (Standard: 4h)
  \item[\verb"-f"] erzwingt das Überschreiben einer eventuell vorhandenen Ausgabedatei.
  \item[\verb"-q"] unterdrückt die Anzeige des Fortschrittbalkens.
\end{description}



\subsection{Funktionsweise}
Es werden beide Tabelle geöffnet und alle Events (primäre Tabelle) durchgegangen. In der Zusatzdatentabelle (sekundäre Tabelle) werden die Zeilen gesucht, die zeitlich vor bzw. nach dem aktuellen Event liegen. Durch lineare Interpolation werden die Zusatzdaten zum Eventzeitpunkt berechnet und dem Event zugeordnet. 

Werden keine Zusatzdaten vor und nach dem aktuellen Event gefunden, oder ist der Abstand zwischen den beiden Zusatzdatenzeilen zu groß (Option \texttt{-x}), so wird das Event verworfen.










\chapter{Ploterzeugung}
\section{plot.py}

\chapter{Webinterface}















\end{document} 